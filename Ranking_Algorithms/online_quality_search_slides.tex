\documentclass{beamer}

% theme
\mode<presentation>
{
\usetheme{Warsaw}

\setbeamercovered{transparent}
}

% packages
\usepackage{multicol}
\usepackage{amsmath}
\usepackage{hyperref}

% setting
\linespread{1.2}

% title page
\title{Consumer Quality Search and Platform Algorithm Design: A Structural Apporach}
\subtitle{Working in Progress}
\author{
Guo Zhang
\thanks{WISE, Xiamen University. Email: zhangguo@stu.xmu.edu.cn. All codes can be found here: \url{https://github.com/xmucpp/online_quality_search/blob/master/consumer_quality_search.ipynb}}
}
\date{This Version: \today}

\begin{document}

\begin{frame}%[plain]
\maketitle
\end{frame}

\begin{frame}[plain]
\frametitle{Contents}
%\begin{multicols}{2}
\tableofcontents[hideallsubsections]
%\end{multicols}
\end{frame}

\section{Introduction}
\begin{frame}
\frametitle{Motivation}
\begin{itemize}
\item A design or improvement of algorithms and mechanisms can produce revenue of millions of dollars for online platforms.
\item An efficient and well-organized method for a certain situation is valuable for business leaders and managers to make data-driven decisions. 
\item The underlying processes that makes them work have not been clarified with structural frameworks for many cases.
\end{itemize}
\end{frame}

\begin{frame}
\frametitle{Theme}
\begin{itemize}
\item Structurally Modeling consumer search based on prices and historical sales, dynamic pricing of sellers and market equilibrium 
on the environment of ranking algorithms based on prices and historical sales on the search engine of Tmall\footnotemark with consideration set approach. 
\footnotetext{Example: \url{https://list.tmall.com/search_product.htm?q=\%BD\%B4\%D3\%CD}}
\item Looking for the optimal weighting of prices and historical sales for ranking algorithms to maximize consumer/seller/total welfare with the structural-form model.
\end{itemize}
\end{frame}

\begin{frame}
\frametitle{Overview}
\begin{itemize}
\item Background of online consumer search and seller pricing
\item Model of consumer search, seller pricing and market equilibrium
\item Estimation of model parameters (doing)
\item Simulation to find optimal weighting of ranking algorithms (to do)
\end{itemize}
\end{frame}

\begin{frame}
\frametitle{Literatures Review}
\begin{itemize}
  \item Search friction and price dispersion
  \item Limited consumer search
  \item Search obfuscation
  \item Reputation Mechanism
\end{itemize}
\end{frame}

\begin{frame}
\frametitle{Literatures Review - Framework}
\begin{itemize}
\item Levin et.al. (2014): structuring consumer search and platform design for online retailers
\item Fan et.al. (2013): cross-period return to reputation and reputation management for sellers
\end{itemize}
\end{frame}

\begin{frame}
\frametitle{Literatures Review - Search Friction and Price Dispersion}
\begin{itemize}
\item Baye et.al. (2014): price dispersion is persistent and is greater in the market with smaller number of products on online markets
\begin{itemize}
\item Reinganum (1979): positive search costs
\item Spulber (1995): private information
\item Stahl (1989): heterogeneous consumers with different search costs
\end{itemize}
\end{itemize}
\end{frame}

\begin{frame}
\frametitle{Literatures Review - Limited Consumer Search}
\begin{itemize}
\item Goeree (2008): limited consumer information and advertising in PC industry with consideration set approach
\item Levin et.al. (2014): search engine instead
\end{itemize}
\end{frame}

\begin{frame}
\frametitle{Literatures Review - Search Obfusaction}
\begin{itemize}
\item Ellison et.al. (2009): search obfuscation on the environment of search engine
\end{itemize}
\end{frame}

%\begin{frame}
%\frametitle{Literatures Review - Reputation Mechanism}
%\begin{itemize}
%\item 
%\end{itemize}
%\end{frame}


\section{Background}
\subsection{The Organization of Markets with Online Platforms}
\begin{frame}
\frametitle{The Organization of Markets with Online Platforms}
\begin{itemize}
  \item Overcome the limitation of geographic and social barriers
  \item Increase the search costs of users by crowding a large amount of information
  \begin{itemize}
    \item Platforms usually desgin algorithms to display products with higher quality and higher relevance for users in practice.
  \end{itemize}
\end{itemize}
\end{frame}

\subsection{Platform Algorithm Design}
\begin{frame}
\frametitle{Platform Algorithm Design}
\begin{itemize}
 \item Ranking: centralized, often all the results
 \item Recommendation: personalized, often a limited amount of results
\end{itemize}
\end{frame}

\begin{frame}
\frametitle{Different Approaches of Ranking Designs}
\begin{itemize}
\item Sort by sales
\item Sort by prices
\item Sort by date
\item Best Match (mixed methods) 
\end{itemize}
\end{frame}

\subsection{Consumer Online Search}
\begin{frame}
\frametitle{Dimensions of Consumer Online Search}
\begin{itemize}
\item Step 1: search relevant products
\begin{itemize}
\item \textbf{User query} or \textbf{keyword search}
\item Advertising (i.e. home page, social media, video platforms)
\item Recommendation (i.e. home page, item page)
\end{itemize}
\item Step 2: search products with attractive \textbf{prices} and \textbf{qualities} (which we will focus in this paper)
\end{itemize}
\end{frame}

\begin{frame}
\frametitle{Consideration Set Approach}
\begin{itemize}
  \item Ranking algorithms provide consideration set from all available products
  \item Consumers make decision in the consideration set
\end{itemize}
\end{frame}

\subsection{Stategies and Decisions of Sellers}
\begin{frame}
\frametitle{Stategies and Decisions of Sellers}
\begin{itemize}
  \item Make strategies on prices, quality, relevance, reputation, etc.
  \item Search obfuscation
\end{itemize}
\end{frame}

\section{Model}
\begin{frame} 
\frametitle{Consumer Utility}
\begin{itemize}
\item Utility of consumer j for product i at period t: 
$$u_{ijt}=\alpha_0 + \alpha_p p_{jt} + \alpha_q ln(q_{j,t-1}) +\varepsilon_{ijt}$$
\item Choice probability given consideration set by logit discrete choice model:
$$
P(j|j\in L_{it}) = \frac{\exp(\alpha_0+\alpha_1 p_{jt}+\alpha_q ln(q_{j,t-1}))}{\sum_{k \in L} \exp(\alpha_0+\alpha_p p_{kt}+\alpha_q ln(q_{k,t-1}))}
$$
\end{itemize}
\end{frame}

\begin{frame}[allowframebreaks]
\frametitle{Consideration Set}
\begin{itemize}
\item Sampling weight for consumer j at period t with Wallenius' non-central hypergeometric distribution:
\begin{align*}
w_{jt}  = &\exp[-\gamma_p (\frac{p_{jt}-\min_{k \in J_t}(p_{kt})}{std_{k \in J_t}}) \\
&+ \gamma_q (\frac{q_{jt}-\min_{k \in J_t}(ln(q_{kt}))}{std_{k \in J_t}(ln(q_{kt}))})  \\
]
\end{align*}
\item The probability of product $j$ in the consideration set at period t:
$$
P(j \in L_t | j \in J_t) = 1 - \Pi_{k=1}^{l}(1-p_{kt})
$$
where
$$
P(j_{lt} \in L_t | j_{1t} \in J_t) = \frac{w_{j_l t}}{\sum_{k \in J_t}(w_{kt})}
$$
\end{itemize}
\end{frame}

\begin{frame}
\frametitle{Market Demand}
\begin{itemize}
\item Choice probability of product j at period t:
$$
D(p_{jt}) = \sum_{L_{it}: L_{it}\in J_t} P(j|j\in L_{it})P(j \in L_{it}|j\in J_t)
$$
\end{itemize}
\end{frame}

\begin{frame}
\frametitle{Seller Pricing and Equilibrium}
\begin{itemize}
\item Profit of seller/product j in each period:
$$
\pi_{jt}  = (p_{jt}-c_{j})D_{jt}(p_{jt})
$$
\item One-Period:
$$
\max_{p_j} \pi_{jt}
$$
\item Muti-Period:
$$
\max_{p_{jt}} \Pi_{j}=\max_{p_jt} \sum_{t=0}^{\infty}\beta^t\pi_{jt}
$$
\end{itemize}
\end{frame}


\section{Data}
\begin{frame}[allowframebreaks]
\frametitle{Data Source}
\begin{itemize}
\item Source: results given keywords on the product search engine of Tmall collected by China's Prices Project\footnotemark
\footnotetext{\tiny{China's Prices Project is an academic initiative and technology firm working on economic and business big data found by Guo Zhang. A data warehouse of price data from the search page of Tmall and JD is constructing with big data technology.}}
\item Time Period: daily data from May 2016 - Present
\item Variables: price, monthly sales, comment number, name of goods, URL of product page, name of shop, name of shop page, rank method, scrape time, category information, page number, rank order, etc.
\item Observations:  more than 1 billion
\end{itemize}
\end{frame}

\begin{frame}
\frametitle{Data Sample}
\begin{itemize}
  \item First 3 pages(117 results) on search result pages displayed on APPs of Tmall and JD from Jan 2017 to now
  \item One day (2017-04-01) of data with keyword "apple" is selected in this version
\end{itemize}
\end{frame}

\begin{frame}
\frametitle{Summarized Statistics}
\begin{table}[!hbp]
\centering
\begin{tabular}{|c|c|}
\hline
\textbf{Listings Data} & \textbf{Values} \\
\hline 
Number of Listings Number of Sellers & 117 \\
\hline
\% of Sellers with $>$ 1  & 25.64\% \\
\hline
Listing Mean List Price & 57.83  \\
\hline 
Median List Price & 46.8  \\
\hline
Standard Deviation of List Price & 44.86 \\
\hline
\end{tabular}
\caption{\label{summary}: Summarized Statistics}
\end{table}
\end{frame}

\begin{frame}
\frametitle{Correlations}
\begin{table}[!hbp]
\centering
\begin{tabular}{lllrl}
       func &            dep & indep &      corr &      p-value \\
       \hline
   pearsonr &          price &  rank &  0.048838 &     0.601044 \\
   pearsonr &     sales\_last &  rank & -0.418686 &  0.000003*** \\
   pearsonr &  comments\_last &  rank & -0.385328 &  0.000018*** \\
  spearmanr &          price &  rank &  0.064618 &     0.488828 \\
  spearmanr &     sales\_last &  rank & -0.571240 &  0.000000*** \\
  spearmanr &  comments\_last &  rank & -0.454517 &  0.000000*** \\
 kendalltau &          price &  rank &  0.040969 &     0.515410 \\
 kendalltau &     sales\_last &  rank & -0.416305 &  0.000000*** \\
 kendalltau &  comments\_last &  rank & -0.326366 &  0.000000*** \\
\end{tabular}
\caption{\label{correlation}: Correlations}
\end{table}
\end{frame}

\section{Estimation}

\begin{frame}
\frametitle{Steps of Estimation}
\begin{enumerate}
\item Demand parameters: parameters of consumer utility $\alpha$
\item Platform parameters: weighting of the ranking algorithm $\gamma$
\item Seller parameters: marginal costs of products $c$
\end{enumerate}
\end{frame}

\begin{frame}
\frametitle{Note for This Version}
\begin{itemize}
\item Accumulated comment number is also considered as a proxy of product quality, which will be excluded for the poor performance in the describe statistics and model estimation.
\item The scale of sales and comment is set as log form to make the estimation work.
\end{itemize}
\end{frame}

\begin{frame}
\frametitle{Estimation of Demand Parameters}
\begin{itemize}
\item Observed consideration set defined as the first 10 products by ranking algorithms
\item MLE directly by mutinomial logit discrete choice model 
\begin{align*}
u_{ij} &= 3.29718207^{-6} -1.31976519^{-2} price_{t} \\
&+  8.79789875^{-1} log(sales_{t-1})+ -8.01401981^{-2} log(comment_{t-1})
\end{align*}
\end{itemize}
\end{frame}

\begin{frame}
\frametitle{Estimation of Platform Parameters}
\begin{itemize}
\item All available products defined as all scraped observations
\item Maximum simulated likelihood estimation
\begin{align*}
w_{jt}  = &\exp[-1.20910197 (\frac{p_{jt}-\min_{k \in J_t}(p_{kt})}{std_{k \in J_t}}) \\
&+ 0.39808008 (\frac{q_{jt}-\min_{k \in J_t}(log(q_{kt}))}{std_{k \in J_t}(log(q_{kt}))})  \\
&+ 0.59927273 (\frac{m_{jt}-\min_{k \in J_t}(log(m_{kt}))}{std_{k \in J_t}(log(m_{kt}))})  
]
\end{align*}
\end{itemize}
\end{frame}

\begin{frame}
\frametitle{Estimation of Seller Parameters}
\begin{itemize}
\item Solving the optimization of maximizing profits of seller:
$$
c_j = p_j + \frac{D_{jt}(p_j)}{D'_{jt}}
$$
\end{itemize}
\end{frame}

\section{Simulation}
\begin{frame}
\frametitle{Simulation Methods}
\begin{itemize}
\item Calculate the total surplus as a function of algorithm parameters given other estimated parameters
\item Change $\gamma$, find a solution that maximize total surplus
\end{itemize}
\end{frame}

\section{Discussions}
\begin{frame}
\frametitle{Next Step}
\begin{itemize}
\item Extend the model from static to dynamic
\item Improve the sample data
\item Relax the assumption that qualities of goods are given / Allow entry and exit of sellers 
\end{itemize}
\end{frame}

\begin{frame}
\frametitle{Further Work}
\begin{itemize}
\item Competition for product with horizontal differentation on one search results
\item Obfuscation on search keywords by sellers
\item Environment of recommender system 
\end{itemize}
\end{frame}


\begin{frame}[plain]
\frametitle{Contents}
%\begin{multicols}{2}
\tableofcontents[hideallsubsections]
%\end{multicols}
\end{frame}

\end{document}